% This is "sig-alternate.tex" V2.1 April 2013
% This file should be compiled with V2.5 of "sig-alternate.cls" May 2012
%
% This example file demonstrates the use of the 'sig-alternate.cls'
% V2.5 LaTeX2e document class file. It is for those submitting
% articles to ACM Conference Proceedings WHO DO NOT WISH TO
% STRICTLY ADHERE TO THE SIGS (PUBS-BOARD-ENDORSED) STYLE.
% The 'sig-alternate.cls' file will produce a similar-looking,
% albeit, 'tighter' paper resulting in, invariably, fewer pages.
%
% ----------------------------------------------------------------------------------------------------------------
% This .tex file (and associated .cls V2.5) produces:
%       1) The Permission Statement
%       2) The Conference (location) Info information
%       3) The Copyright Line with ACM data
%       4) NO page numbers
%
% as against the acm_proc_article-sp.cls file which
% DOES NOT produce 1) thru' 3) above.
%
% Using 'sig-alternate.cls' you have control, however, from within
% the source .tex file, over both the CopyrightYear
% (defaulted to 200X) and the ACM Copyright Data
% (defaulted to X-XXXXX-XX-X/XX/XX).
% e.g.
% \CopyrightYear{2007} will cause 2007 to appear in the copyright line.
% \crdata{0-12345-67-8/90/12} will cause 0-12345-67-8/90/12 to appear in the copyright line.
%
% ---------------------------------------------------------------------------------------------------------------
% This .tex source is an example which *does* use
% the .bib file (from which the .bbl file % is produced).
% REMEMBER HOWEVER: After having produced the .bbl file,
% and prior to final submission, you *NEED* to 'insert'
% your .bbl file into your source .tex file so as to provide
% ONE 'self-contained' source file.
%
% ================= IF YOU HAVE QUESTIONS =======================
% Questions regarding the SIGS styles, SIGS policies and
% procedures, Conferences etc. should be sent to
% Adrienne Griscti (griscti@acm.org)
%
% Technical questions _only_ to
% Gerald Murray (murray@hq.acm.org)
% ===============================================================
%
% For tracking purposes - this is V2.0 - May 2012

\documentclass{sig-alternate-05-2015}

%%%%%%%%%%%% DEFINE PACKAGES HERE!!
\usepackage{algorithmic}
\usepackage{algorithm}
%\usepackage{minted}

\begin{document}

% Copyright
%\setcopyright{acmcopyright}
%\setcopyright{acmlicensed}
\setcopyright{rightsretained}
%\setcopyright{usgov}
%\setcopyright{usgovmixed}
%\setcopyright{cagov}
%\setcopyright{cagovmixed}


% DOI
\doi{XX.XXX/XXX_X}

% ISBN
\isbn{X-XXXXX-XX-X/XX/XX}

%Conference
\conferenceinfo{Bio331}{Fall 2016, Reed College, Portland, OR}

%\acmPrice{\$15.00}

%
% --- Author Metadata here ---
%\conferenceinfo{WOODSTOCK}{'97 El Paso, Texas USA}
%\CopyrightYear{2007} % Allows default copyright year (20XX) to be over-ridden - IF NEED BE.
%\crdata{0-12345-67-8/90/01}  % Allows default copyright data (0-89791-88-6/97/05) to be over-ridden - IF NEED BE.
% --- End of Author Metadata ---

\title{Introduction to Pseudocode and Math}
\date{\today}
%
% You need the command \numberofauthors to handle the 'placement
% and alignment' of the authors beneath the title.
%
% For aesthetic reasons, we recommend 'three authors at a time'
% i.e. three 'name/affiliation blocks' be placed beneath the title.
%
% NOTE: You are NOT restricted in how many 'rows' of
% "name/affiliations" may appear. We just ask that you restrict
% the number of 'columns' to three.
%
% Because of the available 'opening page real-estate'
% we ask you to refrain from putting more than six authors
% (two rows with three columns) beneath the article title.
% More than six makes the first-page appear very cluttered indeed.
%
% Use the \alignauthor commands to handle the names
% and affiliations for an 'aesthetic maximum' of six authors.
% Add names, affiliations, addresses for
% the seventh etc. author(s) as the argument for the
% \additionalauthors command.
% These 'additional authors' will be output/set for you
% without further effort on your part as the last section in
% the body of your article BEFORE References or any Appendices.

\numberofauthors{1} %  in this sample file, there are a *total*
% of EIGHT authors. SIX appear on the 'first-page' (for formatting
% reasons) and the remaining two appear in the \additionalauthors section.
%
\author{
% You can go ahead and credit any number of authors here,
% e.g. one 'row of three' or two rows (consisting of one row of three
% and a second row of one, two or three).
%
% The command \alignauthor (no curly braces needed) should
% precede each author name, affiliation/snail-mail address and
% e-mail address. Additionally, tag each line of
% affiliation/address with \affaddr, and tag the
% e-mail address with \email.
%
% 1st. author
\alignauthor
Anna Ritz\\
       \affaddr{Biology Department, Reed College}\\
       \affaddr{Portland, Oregon}\\
       \email{aritz@reed.edu}\\\date{\today}
}

\maketitle

\begin{abstract}
This ACM-style template describes how to typeset pseudocode as well as write common mathematical symbols.  \textbf{Copy this project and start by modifying the title, author, etc.}  There are also very useful URLS on Moodle for more information.
\end{abstract}

\keywords{pseudocode, algorithms, math, LaTeX}


\section{Pseudocode}
The goal is to find the flow-betweenness centrality of all of the nodes in the graph.  To do this I use alorgithm \ref{FBC} to get the normalized amount of flow calculated though each pair of points.  This algrorithm uses the Fodr-Fulkerson method to caluclate individual flows though nodes.  I seperate the flows into four different sections which use different subsets depending on which social group the node in question and to the sorce and target node are from.  First, $c_{total}$ inlcudes all of the possible pairing of source and sink nodes.  Seccond $c_{btwn}$ consists of sorce and target are not from the same socal group.  The next two are if the source and the target are in the same social group: $c_{inter}$ has the node in question in that same social group, $c_{out}$ has the node in question in a different social group. Algorithm \ref{FBC} interates between all unique combinations of sources and targes with $(u,v) = (v,u)$ and at each combination finds the normalized flow for all the nodes using \ref{FF}.  It then addes those flows to the centrality group that they are a part of.  

\begin{algorithm}
\caption{Flow-Betweenness Centrality}
\label{FBC}
\begin{algorithmic}
\STATE \textbf{Inputs:} $G(E,V)$ and edge weights $w$ social groups s
\STATE \textbf{Outputs:} flow-betweenness centrality for nodes
\STATE $c_{inter}(n) \leftarrow 0 \text{ for } n \in E$
\STATE $c_{btwn}(n) \leftarrow 0 \text{ for } n \in E$
\STATE $c_{out}(n) \leftarrow 0 \text{ for } n \in E$
\FOR {$k \in V$}
\FOR{$j \in V : (j,k) \in E \text{ and } j < k$}
\STATE $ f(u,v) = \text{Ford-Fulkerson} (G,w,j,k)$
\FOR{n $\in V : n \neq j \text{ and } n \neq k$}
\STATE $c(n) = \sum_{o \in N(n)}\frac{| f(n,o)|}{2 (\sum_{o \in N(k)}|f(k,o)|)} \text{ where } N(n) \text{ are n's neighbors} $
\IF { $s(j) = s(k) \text{ \textbf{and} } s(j) = s(n)$}
\STATE $c_{inter} = c_{inter} + c$
\ELSIF {$s(j) = s(k)$}
\STATE $c_{out} = c_{out} + c$
\ELSE 
\STATE $c_{btwn} = c_{btwn} + c$
\ENDIF
\STATE $c_{total} = c_{total} + c$
\ENDFOR
\ENDFOR
\ENDFOR
\RETURN $c_{total}, c_{inter}, c_{btwn}, c_{out}$
\end{algorithmic}
\end{algorithm}



The Ford-Fulkerson method (Algorithm \ref{FF}) works by simulating putting flow throught the graph.  It does this by keeping track of a residual network ($G_{f}$) which is a representation of how much more flow can go through each of the edges.  The algorithm goes until there is not path from the source to the sink where all of the edges $>$ 0 in the residual graph.  In other words this is untill there can be no more flow to the target from the source.  Each iteration the algorithm find a path using depth first search on $G_{f}$, then it adds the minimun edge$ c_{f}$ value to each of the flows and the recalculates $G_{f}$.  This is like sending the most possible flow though that path which is restricted by edge with the lowest flow.  Once there are no more paths in $G_{f}$ from the source to the target it will return the flows for each of the edges in G.


\begin{algorithm}[t]
\caption{Ford-Fulkerson method}
\label{FF}
\begin{algorithmic}
\STATE $\text{ \textbf{Inputs:} A network } G = (V, E) \text{ with flow capasity } c, \text{ source } s, \text{ and target } t$ 
\STATE $ \text{ \textbf{Outputs:} Flows } f(u, v) \text{ for all }(u, v) \in E \text{ between }s \text{ and }t$
\STATE $f(u, v) \leftarrow 0 \text{ for all edges } (u, v)$
\WHILE{  $\text{there exits a path } p_{st} \text{ in } G_{f} : c_{f} (u,v) > 0 \text{ for all edges } (u, v) \in p$}
\STATE find $c_{f}(p) = \min (c_{f}:(u, v) \in P)$
\FOR{each edge $(u,v) \in p$}
\STATE  $f(u, v) \leftarrow f(u,v) + c_{f}(p)$
\STATE  $f(v, u) \leftarrow f(v,u) - c_{f}(p)$
\ENDFOR
\ENDWHILE
\RETURN $f(u,v)$
\end{algorithmic}
\end{algorithm}






\end{document}
