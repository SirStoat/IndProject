% This is "sig-alternate.tex" V2.1 April 2013
% This file should be compiled with V2.5 of "sig-alternate.cls" May 2012
%
% This example file demonstrates the use of the 'sig-alternate.cls'
% V2.5 LaTeX2e document class file. It is for those submitting
% articles to ACM Conference Proceedings WHO DO NOT WISH TO
% STRICTLY ADHERE TO THE SIGS (PUBS-BOARD-ENDORSED) STYLE.
% The 'sig-alternate.cls' file will produce a similar-looking,
% albeit, 'tighter' paper resulting in, invariably, fewer pages.
%
% ----------------------------------------------------------------------------------------------------------------
% This .tex file (and associated .cls V2.5) produces:
%       1) The Permission Statement
%       2) The Conference (location) Info information
%       3) The Copyright Line with ACM data
%       4) NO page numbers
%
% as against the acm_proc_article-sp.cls file which
% DOES NOT produce 1) thru' 3) above.
%
% Using 'sig-alternate.cls' you have control, however, from within
% the source .tex file, over both the CopyrightYear
% (defaulted to 200X) and the ACM Copyright Data
% (defaulted to X-XXXXX-XX-X/XX/XX).
% e.g.
% \CopyrightYear{2007} will cause 2007 to appear in the copyright line.
% \crdata{0-12345-67-8/90/12} will cause 0-12345-67-8/90/12 to appear in the copyright line.
%
% ---------------------------------------------------------------------------------------------------------------
% This .tex source is an example which *does* use
% the .bib file (from which the .bbl file % is produced).
% REMEMBER HOWEVER: After having produced the .bbl file,
% and prior to final submission, you *NEED* to 'insert'
% your .bbl file into your source .tex file so as to provide
% ONE 'self-contained' source file.
%
% ================= IF YOU HAVE QUESTIONS =======================
% Questions regarding the SIGS styles, SIGS policies and
% procedures, Conferences etc. should be sent to
% Adrienne Griscti (griscti@acm.org)
%
% Technical questions _only_ to
% Gerald Murray (murray@hq.acm.org)
% ===============================================================
%
% For tracking purposes - this is V2.0 - May 2012

\documentclass{sig-alternate-05-2015}

%%%%%%%%%%%% DEFINE PACKAGES HERE!!
\usepackage{algorithmic}
\usepackage{algorithm}
\usepackage{minted}

\begin{document}

% Copyright
%\setcopyright{acmcopyright}
%\setcopyright{acmlicensed}
\setcopyright{rightsretained}
%\setcopyright{usgov}
%\setcopyright{usgovmixed}
%\setcopyright{cagov}
%\setcopyright{cagovmixed}


% DOI
\doi{XX.XXX/XXX_X}

% ISBN
\isbn{X-XXXXX-XX-X/XX/XX}

%Conference
\conferenceinfo{Bio331}{Fall 2016, Reed College, Portland, OR}

%\acmPrice{\$15.00}

%
% --- Author Metadata here ---
%\conferenceinfo{WOODSTOCK}{'97 El Paso, Texas USA}
%\CopyrightYear{2007} % Allows default copyright year (20XX) to be over-ridden - IF NEED BE.
%\crdata{0-12345-67-8/90/01}  % Allows default copyright data (0-89791-88-6/97/05) to be over-ridden - IF NEED BE.
% --- End of Author Metadata ---

\title{Introduction to Pseudocode and Math}
\date{\today}
%
% You need the command \numberofauthors to handle the 'placement
% and alignment' of the authors beneath the title.
%
% For aesthetic reasons, we recommend 'three authors at a time'
% i.e. three 'name/affiliation blocks' be placed beneath the title.
%
% NOTE: You are NOT restricted in how many 'rows' of
% "name/affiliations" may appear. We just ask that you restrict
% the number of 'columns' to three.
%
% Because of the available 'opening page real-estate'
% we ask you to refrain from putting more than six authors
% (two rows with three columns) beneath the article title.
% More than six makes the first-page appear very cluttered indeed.
%
% Use the \alignauthor commands to handle the names
% and affiliations for an 'aesthetic maximum' of six authors.
% Add names, affiliations, addresses for
% the seventh etc. author(s) as the argument for the
% \additionalauthors command.
% These 'additional authors' will be output/set for you
% without further effort on your part as the last section in
% the body of your article BEFORE References or any Appendices.

\numberofauthors{1} %  in this sample file, there are a *total*
% of EIGHT authors. SIX appear on the 'first-page' (for formatting
% reasons) and the remaining two appear in the \additionalauthors section.
%
\author{
% You can go ahead and credit any number of authors here,
% e.g. one 'row of three' or two rows (consisting of one row of three
% and a second row of one, two or three).
%
% The command \alignauthor (no curly braces needed) should
% precede each author name, affiliation/snail-mail address and
% e-mail address. Additionally, tag each line of
% affiliation/address with \affaddr, and tag the
% e-mail address with \email.
%
% 1st. author
\alignauthor
Anna Ritz\\
       \affaddr{Biology Department, Reed College}\\
       \affaddr{Portland, Oregon}\\
       \email{aritz@reed.edu}\\\date{\today}
}

\maketitle

\begin{abstract}
This ACM-style template describes how to typeset pseudocode as well as write common mathematical symbols.  \textbf{Copy this project and start by modifying the title, author, etc.}  There are also very useful URLS on Moodle for more information.
\end{abstract}

\keywords{pseudocode, algorithms, math, LaTeX}


\section{Typesetting Math}

LaTeX makes typesetting math \textit{very} easy.  You can use dollar signs to typeset variables and short mathematical expressions within the paragraph.  For example, supposed we have three variables $x$, $y$, and $z$; we can define a linear equation as $3x+2y-z=10$.  Other examples of math are shown in Table~\ref{math}.  Use curly braces to group multiple characters in subscripts or superscripts.

%In the line below, [h] means "here", [t] means "top of page", [b] means "bottom of page"
\begin{table}[h]  
\centering
\begin{tabular}{r|l}  % r means "right adjust", l means "left adjust", pipe denotes vertical line.
LaTeX & Math \\ \hline
\$x\$ & $x$ \\
\$S\_i\$ & $S_i$ \\
\$$\backslash$sum\_\{i=1\}\^{}\{20\} S\_i\$ & $\sum_{i=1}^{20} S_i$ \\
\$$\backslash$max\_x f(x)\$ & $\max_x f(x)$ \\
\$A $\backslash$cup B\$ & $A \cup B$ \\
\$$\backslash$bigcup\_i S\_i\$ & $\bigcup_i S_i$ \\
\$x $\backslash$text\{ for all \} x $\backslash$in X\$ & $x \text{ for all } x \in X$ \\
\end{tabular}\caption{Common math syntax.} 
\label{math}
\end{table}

You can offset equations with double dollar signs, such as 
$$x=y+z-w.$$
You can also use multi-line equations that are aligned with the \texttt{align} package:
\begin{align}
x &= \prod_{u \in U} |X_u| \\
x_2 & \geq y_0 + y_1 + \sum_{j=1}^5 z_j \label{line2}
\end{align}
Here, the equations will be numbered, and you can refer to them with labels, e.g., Equation~\eqref{line2} is the second equation.

\section{Pseudocode}

There are many algorithms packages in LaTeX; here, I'll describe the \texttt{algorithmic} package, which I've used for this course.  You'll also want to be able to number these algorithms (like Tables or Figures); this can be done with the \texttt{algorithm} package.  At the top of this file (the .tex), you will see two \texttt{\\usepackage} commands that import these packages.

\subsection{The \texttt{algorithmic} Package}

Consider the following snippet of code.

\begin{verbatim}
\begin{algorithmic}
\STATE $i=1$
\WHILE{$i < 10$}
\STATE $i=i+1$
\ENDWHILE
\RETURN i
\end{algorithmic}
\end{verbatim}

\noindent This code in LaTeX produces the formatted pseudocode.

\begin{algorithmic}
\STATE $i=1$
\WHILE{$i < 10$}
\STATE $i=i+1$
\ENDWHILE
\RETURN i
\end{algorithmic}
\vspace{.2in}

\noindent The syntax for different statements is below.  The text may be words or math.
\begin{itemize}
\item $\backslash$STATE $<$text$>$
\item $\backslash$RETURN $<$text$>$
\item $\backslash$IF\{$<$condition$>$\} $<$text$>$ $\backslash$ENDIF
\item $\backslash$WHILE\{$<$condition$>$\} $<$text$>$ $\backslash$ENDWHILE
\item $\backslash$FOR\{$<$condition$>$\} $<$text$>$ $\backslash$ENDFOR
\item $\backslash$AND, $\backslash$OR, $\backslash$NOT: logical connectives
\end{itemize}

\subsection{The \texttt{algorithm} Environment}

To add a caption, name, or label to your algorithm, you can use the \texttt{algorithm} environment.  It will move as you modify text to make a decent layout.  The LaTeX below produces the nicely-formatted algorithm, and you can refer to it as Algorithm~\ref{alg-label}.

\begin{verbatim}
\begin{algorithm}
\caption{An example algorithm.}
\label{alg-label}
\begin{algorithmic}
\STATE $i=1$
\WHILE{$i < 10$}
\STATE $i=i+1$
\ENDWHILE
\RETURN i
\end{algorithmic}
\end{algorithm}
\end{verbatim}

\begin{algorithm}
\caption{An example algorithm.}
\label{alg-label}
\begin{algorithmic}
\STATE $i=1$
\WHILE{$i < 10$}
\STATE $i=i+1$
\ENDWHILE
\RETURN i
\end{algorithmic}
\end{algorithm}

\subsection{Typesetting Code}

Suppose you want to write a Python function definition.  You can color-format the code using the \texttt{minted} package:

\begin{verbatim}
\begin{minted}{python}
def simple_add(x,y):
  z = x + y
  return z
\end{minted}
\end{verbatim}

\noindent The above lines of LaTeX produces:
\begin{minted}{python}
def simple_add(x,y):
  z = x + y
  return z
\end{minted}

\section{Pseudocode Writeup}

Write a description of the code you are writing.  
\begin{enumerate}
\item Write pseudocode (a mix of code, math, and concise terms) that describes a major component of your project.
\item Write a summary of the pseudocode that provides intuition.
\end{enumerate}

\noindent If you have multiple parts of your project, in addition to the pseudocode you can make an \textbf{overview figure} that shows how the parts are connected.  

\paragraph{Evaluation}  I will assess and provide feedback on the \textbf{notation and syntax} of your pseudocode and the \textbf{clarity} of your summary. I will also provide feedback on an overview figure if it's submitted.

\paragraph{Handin Instructions} Hand in the PDF of your LaTeX document describing the pseudocode by Tuesday at 5pm.

\end{document}
